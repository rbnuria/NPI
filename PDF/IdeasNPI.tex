%%%%%%%%%%%%%%%%%%%%%%%%%%%%%%%%%%%%%%%%%%%%%%%%%%%%%%%%%%%%%%%%%%%%%%%%%%%%%%%%%%%%%%%%%%%%%%%%%%%%%%
% Plantilla básica de Latex en Español.
%
% Autor: Andrés Herrera Poyatos (https://github.com/andreshp) 
%
% Es una plantilla básica para redactar documentos. Utiliza el paquete fancyhdr para darle un
% estilo moderno pero serio.
%
% La plantilla se encuentra adaptada al español.
%
%%%%%%%%%%%%%%%%%%%%%%%%%%%%%%%%%%%%%%%%%%%%%%%%%%%%%%%%%%%%%%%%%%%%%%%%%%%%%%%%%%%%%%%%%%%%%%%%%%%%%%

%-----------------------------------------------------------------------------------------------------
%	INCLUSIÓN DE PAQUETES BÁSICOS
%-----------------------------------------------------------------------------------------------------

\documentclass{article}

\usepackage{lipsum}                     % Texto dummy. Quitar en el documento final.

%-----------------------------------------------------------------------------------------------------
%	SELECCIÓN DEL LENGUAJE
%-----------------------------------------------------------------------------------------------------

% Paquetes para adaptar Látex al Español:
\usepackage[spanish,es-noquoting, es-tabla, es-lcroman]{babel} % Cambia 
\usepackage[utf8]{inputenc}                                    % Permite los acentos.
\selectlanguage{spanish}                                       % Selecciono como lenguaje el Español.

%-----------------------------------------------------------------------------------------------------
%	SELECCIÓN DE LA FUENTE
%-----------------------------------------------------------------------------------------------------

% Fuente utilizada.
\usepackage{courier}                    % Fuente Courier.
\usepackage{microtype}                  % Mejora la letra final de cara al lector.

%-----------------------------------------------------------------------------------------------------
%	ESTILO DE PÁGINA
%-----------------------------------------------------------------------------------------------------

% Paquetes para el diseño de página:
\usepackage{fancyhdr}               % Utilizado para hacer títulos propios.
\usepackage{lastpage}               % Referencia a la última página. Utilizado para el pie de página.
\usepackage{extramarks}             % Marcas extras. Utilizado en pie de página y cabecera.
\usepackage[parfill]{parskip}       % Crea una nueva línea entre párrafos.
\usepackage{geometry}               % Asigna la "geometría" de las páginas.

% Se elige el estilo fancy y márgenes de 3 centímetros.
\pagestyle{fancy}
\geometry{left=3cm,right=3cm,top=3cm,bottom=3cm,headheight=1cm,headsep=0.5cm} % Márgenes y cabecera.
% Se limpia la cabecera y el pie de página para poder rehacerlos luego.
\fancyhf{}

% Espacios en el documento:
\linespread{1.1}                        % Espacio entre líneas.
\setlength\parindent{0pt}               % Selecciona la indentación para cada inicio de párrafo.

% Cabecera del documento. Se ajusta la línea de la cabecera.
\renewcommand\headrule{
	\begin{minipage}{1\textwidth}
	    \hrule width \hsize 
	\end{minipage}
}

% Texto de la cabecera:
\lhead{\docauthor}                          % Parte izquierda.
\chead{}                                    % Centro.
\rhead{\subject \ - \doctitle}              % Parte derecha.

% Pie de página del documento. Se ajusta la línea del pie de página.
\renewcommand\footrule{                                 
\begin{minipage}{1\textwidth}
    \hrule width \hsize   
\end{minipage}\par
}

\lfoot{}                                                 % Parte izquierda.
\cfoot{}                                                 % Centro.
\rfoot{Página\ \thepage\ de\ \protect\pageref{LastPage}} % Parte derecha.


%-----------------------------------------------------------------------------------------------------
%	PORTADA
%-----------------------------------------------------------------------------------------------------

% Elija uno de los siguientes formatos.
% No olvide incluir los archivos .sty asociados en el directorio del documento.
%\usepackage{title1}
%\usepackage{title2}
%\usepackage{title3}

%-----------------------------------------------------------------------------------------------------
%	TÍTULO, AUTOR Y OTROS DATOS DEL DOCUMENTO
%-----------------------------------------------------------------------------------------------------

% Título del documento.
\newcommand{\doctitle}{Ideas proyecto}
% Subtítulo.
%\newcommand{\docsubtitle}{Ideas proyecto}
% Fecha.
%\newcommand{\docdate}{1 \ de \ Enero \ de \ 2015}
% Asignatura.
\newcommand{\subject}{Nuevos Paradigmas de Interacción}
% Autor.
%\newcommand{\docauthor}{Antonio Rafael Moya Martín-Castaño\\Nuria Rodríguez Barroso \\Elena Romero Contreras \\Juan Luis Suárez Díaz}
\newcommand{\docaddress}{Universidad de Granada}
%\newcommand{\docemail}{andreshp9@gmail.com}

%-----------------------------------------------------------------------------------------------------
%	RESUMEN
%-------------------------------					----------------------------------------------------------------------

% Resumen del documento. Va en la portada.
% Puedes también dejarlo vacío, en cuyo caso no aparece en la portada.
%\newcommand{\docabstract}{}
\newcommand{\docabstract}{En este texto puedes incluir un resumen del documento. Este informa al lector sobre el contenido del texto, indicando el objetivo del mismo y qué se puede aprender de él.}

\begin{document}

%\maketitle

%-----------------------------------------------------------------------------------------------------
%	ÍNDICE
%-----------------------------------------------------------------------------------------------------

% Profundidad del Índice:
%\setcounter{tocdepth}{1}

%\newpage
%\tableofcontents
%\newpage

%-----------------------------------------------------------------------------------------------------
%	SECCIÓN 1
%-----------------------------------------------------------------------------------------------------

\section{Sensores de dispositivo móvil}

\begin{enumerate}
	\item \textbf{Reconocimiento de las entradas con NFC Tags:} La idea sería controlar la entrada al museo con la compra anticipada de entradas por internet. Esta compra proporcionaría al usuario la opción de entrar al museo simplemente escaneando la NFC Tag a la entrada. Además, si has comprado una entrada para un día concreto podrías salir y volver a entrar en el museo sin problemas si quisieras. 
	
	Se puede generalizar esta idea a un museo más grande con diferentes salas bien diferenciadas. Así, existirían diferentes tipos de entradas: museo completo o ciertas salas, y el acceso a cada sala del museo se podría realizar escaneando la NFC Tag correspondiente a la entrada comprada. De esta manera se ahorrarían las colas a la entrada de cada sala.
	
	Además podría controlarse el Wifi del museo para conectar automáticamente los móviles a su red Wifi también con el NFC.

	\item \textbf{Implementación de un juego multijugador:} El juego consisitiría en la búsqueda de ciertos objetos o lugares por el museo. Para controlar que se ha llegado al sitio, podría hacerse de varias maneras: apuntando con la cámara al objeto en cuestión y que la aplicación lo reconozca, que el móvil detectara con la localización que ha llegado a ese lugar, escaneando alguna NFC Tag o contestando a alguna pregunta propuesta por el juego cuya respuesta se encuentre en dicho lugar. Adicionalmente, el juego podrá proporcionar pistas al jugador haciendo uso de la geolocalización. Para la parte de multijugador, ambos jugadores se podrán conectar por Bluetooth para competir. En caso de que el museo sea demasiado grande, la opción multijugador se puede desarrollar solo en una sala del mismo.

	\item \textbf{Uso de NFC Tags para reconocimiento de exposiciones:} Se propone utilizar las NFC Tags para el reconocimiento de ciertas exposiciones. Así, cuando el usuario se acerce a algún objeto con descripción, podrá acercar su teléfono reconociendo la NFC Tags obteniendo en su móvil información sobre dicho objeto: de forma escrita, en modo de audioguía, etcétera. Esto puede ser útil para ofrecer la información en muchos idiomas sin tener que escribir en una etiqueta la misma información en diferentes idiomas. 

	\item \textbf{Uso del podómetro:} para contabilizar los pasos andados por una persona desde que entró al museo. Así, si vemos que se contabilizan un número alto de pasos, le proponemos que se dirija en siguiente lugar a alguna de las salas que cuente con una exposición en la que permanecer sentado.

	\item \textbf{Uso de interacción multitáctil} a lo largo del desarrollo de la aplicación: por ejemplo, cuando estemos leyendo una explicación muy larga, podremos pasar página con los dedos, hacer zoom, etc.

	\item \textbf{Uso del giroscopio para ahorro de energía} en el dispositivo, de manera que baste con bajar el móvil para apagar la pantalla y bajar el consumo de la aplicación. Además, el procedimiento sería bastante cómodo, pues solo hay que dejar descansar el brazo con el que se sujeta el móvil para la suspensión de la aplicación, que volvería a activarse en el momento en que se levantara el brazo, y en el mismo estado en el que se había dejado antes de la suspensión.

	\item \textbf{Regular el volumen} de la aplicación en función del sonido detectado en el ambiente. Si en algún momento la aplicación tuviera que dar alguna explicación oral, podría determinar a qué volumen darla para ser correctamente escuchada a partir del volumen de sonido detectado. Esto podría extenderse para sincronizarse con otros elementos del museo (la música de fondo, algún panel parlante, etc), para que redujeran su volumen en el momento de una explicación.	
	
	\item Implementación del reproductor multimedia para la audioguía haciendo uso de los movimientos del móvil para pasar de pista, pausa, play, etc.
\end{enumerate}

\section{Interacción gestual}

Utilizaremos Kinect para la interacción con nuestro personaje. Se podrán realizar las siguientes funciones:
\begin{enumerate}
	\item \textbf{Selección de objetos en la pantalla}: como puede ser el idioma, o para proponer varias opciones para mudos. Ej: ¿Quieres que te hable de mi historia, lugar, profesión...? Bastará con señalar una de las opciones. El diálogo sería el mismo pero adaptamos el museo a personas con diversidad funcional.

	\item \textbf{Gestos de afirmación y negación:} Cuando el personaje nos haga una pregunta de respuesta afirmativa o negativa, por ejemplo: ¿Quieres que te cuente más?, podremos responderle afirmando o negando mediante un gesto.

	\item \textbf{Pausa del personaje}: Implementación de pausa para hacer que el personaje deje de hablar y para hacer que continue.
	
	\item \textbf{Movimiento/zoom sobre objetos:} Implementación de gestos para mover ciertos objetos que nos mostrará el personaje, de manera que podamos verlos mejor desde distintos ángulos. También se podrá hacer zoom sobre ellos mediante un gesto.
\end{enumerate}


	


\end{document}